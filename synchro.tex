include(`macros.m4')

\pagebreak
\pdfbookmark[0]{process synchronization and communication}{synchro}

\begin{slide}
\sltitle{Contents}
\slidecontents{6}
\end{slide}

%%%%%%%%%%%%%%%%%%%%%%%%%%%%%%%%%%%%%%%%%%%%%%%%%%%%%%%%%%%%%%%%%%%%%%%%%

\pdfbookmark[1]{shared data conflict}{conflict}

\begin{slide}
\sltitle{Problem: conflict while sharing data}
\begin{itemize}
ifdef([[[NOSPELLCHECK]]], [[[
\item \texttt{struct \{ int a, b; \} \emph{shared};}
]]])
\item
\begin{alltt}
for( ; ; ) \{
    \emprg{/* non-atomic operation */
    a = shared.a; b = shared.b;}
    if (a != b) printf("NON-CONSISTENT STATE");
    \emprg{/* non-atomic operation */
    shared.a = val; shared.b = val;}
\} 
\end{alltt}
\item if the cycle is run in 2 processes running in parallel (or threads)
that share the same structure and have different values of the
\texttt{val} variable, it will lead to conflicts due to non-atomic operations.
\end{itemize}
\end{slide}

\label{SYNCHRONIZATION}

\begin{itemize}
\item Operations that can be expressed in C with single statement does not
have to be atomic. e.g. on RISC processors the command \verb.a++.
is typically translated into:

\begin{verbatim}
load reg,[a]
inc reg
store [a],reg
\end{verbatim}

due to the fact that on this architecture numbers cannot be incremented
directly in memory. For such cases there are sets of functions for atomic
arithmetic operations (e.g. \texttt{atomic\_add(3c)} on Solaris) that are
much faster than classic synchronization mechanisms.
More on page \pageref{ATOMIC_ADD}.
\item In general similar problem happens when multiple processes share a 
system resource.
\item \label{RACE_C} example \example{race/race.c}
\end{itemize}

%%%%%

\begin{slide}
\sltitle{Conflict scenario}
ifdef([[[NOSPELLCHECK]]], [[[
\begin{tabular}{rl@{\hspace{2cm}}|c|c|}
]]])
\multicolumn{2}{l}{Processes \emsl{A}\texttt{(val==1)} and
\emsl{B}\texttt{(val==2)}} & \multicolumn{1}{c}{\texttt{a}} &
\multicolumn{1}{c}{\texttt{b}}\\
% \cline{3-4}
1. & initial structure state & ? & ? \\
% \cline{3-4}
2. & process \emsl{A} writes to member \texttt{a} & 1 & ? \\
% \cline{3-4}
3. & process \emsl{B} writes to member \texttt{a} & 2 & ? \\
% \cline{3-4}
4. & process \emsl{B} writes to member \texttt{b} & 2 & 2 \\
% \cline{3-4}
5. & process \emsl{A} writes to member \texttt{b} & 2 & 1 \\
% \cline{3-4}
6. & \multicolumn{1}{l}{\parbox[t]{5cm}{the structure is in inconsistent state
and one of the processes will find out.}}
\end{tabular}
\end{slide}

\begin{itemize}
\item more possibilities:
\begin{enumerate}
\item structure is in inconsistent state, e.g. \texttt{(1, 1)}
\item process \texttt{B} writes \texttt{2} into member \texttt{a}
\item process \texttt{A} reads the value of the structure \texttt{(2, 1)}
earlier than process \texttt{B} writes member \texttt{b}
\end{enumerate}
\item Note that synchronization problems may surface only after the program is
run on multiprocessor(core) machine or on more processors than what is used when
development.  However, given that nowadays you have a few CPUs even on a
low-level laptop, it is easier to test such scenarios than in the past.
\end{itemize}

%%%%%

\pdfbookmark[1]{mutual exclusion}{mutexcl}

\begin{slide}
\sltitle{Solution: mutual process exclusion}
\begin{itemize}
\item it is necessary to ensure atomic operation on the structure, i.e.
while one processes modifies the structure, the other cannot manipulate it. 
\end{itemize}
ifdef([[[NOSPELLCHECK]]], [[[
\begin{tabular}{rl@{\hspace{2cm}}|c|c|}
]]])
\multicolumn{2}{l}{Processes \emsl{A}\texttt{(val==1)} and
\emsl{B}\texttt{(val==2)}} & \multicolumn{1}{c}{\texttt{a}} &
\multicolumn{1}{c}{\texttt{b}}\\
% \cline{3-4}
1. & initial structure state & ? & ? \\
% \cline{3-4}
2. & process \emsl{A} writes to member \texttt{a} & 1 & ? \\
% \cline{3-4}
\setbox0=\hbox{$\left\{\vphantom{\begin{tabular}{c}\hline1\\
\hline1\\ \hline1\\ \hline \end{tabular}}\right.$}
\ht0=0pt\dp0=0pt\box0
3. & process \emsl{B} must wait & 1 & ? \\
% \cline{3-4}
4. & process \emsl{A} writes to member \texttt{b} & 1 & 1 \\
% \cline{3-4}
\setbox0=\hbox{$\left\{\vphantom{\begin{tabular}{c}\hline1\\
\hline1\\ \hline \end{tabular}}\right.$}
\dimen0=\ht0 \advance\dimen0 by \dp0 \dimen0=0.25\dimen0
\ht0=0pt\dp0=0pt \raisebox{-\dimen0}{\box0}%
5. & process \emsl{B} writes to member \texttt{a} & 2 & 1 \\ 
% \cline{3-4}
6. & process \emsl{B} writes to member \texttt{b} & 2 & 2 \\ 
% \cline{3-4}
7. & \multicolumn{1}{l}{the structure is consistent.}\\
\end{tabular}
\end{slide}

\label{CRITICALSECTION}

\begin{itemize}
\item It is necessary to ensure mutual exclusion also when reading,
so that the process that is reading cannot read inconsistent state while
another process is changing it. While writing it is necessary to exclude
all other processes. While reading it is sufficient to exclude only writers.
\item \emsl{\emph{critical section}} is piece of code that can be executed
only in one process (or thread), otherwise the operations can lead to
inconsistent state, e.g. wrongly connected linked list, mismatched database
indexes etc. It is possible to say that critical section is code which accesses
or modifies resource shared with multiple processes (or threads) and therefore
access to such code should be synchronized. Critical section should be as
short as possible to limit contention. The second definition is more generic,
it can also include situations where only one process (or thread) can change
the state however if this is not happening, more processes can read
simultaneously.
\end{itemize}

%%%%%

\begin{slide}
\sltitle{Problem: readers and writers conflict}
\begin{itemize}
\item several processes running in parallel write protocol with operations
to common log file. Each new record is appended to the end of the file.
\item if the record writing operation is not atomic, the contents of multiple
records can be interleaved.
\item only single process can write.
\item other processes read data from the log file.
\item while reading record that is being written inconsistent data is retrieved.
\item while writing, it is not possible to read. If no process is writing,
multiple processes can read simultaneously.
\end{itemize}
\end{slide}

\begin{itemize}
\item 2 situations are permitted: one writer or multiple readers
\item On local disk the synchronization of writers can be archived via the
\texttt{O\_APPEND} flag, however this is not going to work on network file
systems such as NFS or in case it is necessary to perform multiple
\texttt{write()} operations for single record. Moreover this does not solve
the situation of readers -- it is still possible to read while writing.
\end{itemize}

%%%%%

\begin{slide}
\sltitle{Solution: file locking}
\begin{itemize}
\item writer process locks the file for writing. Other processes
(readers and writers) cannot work with the file and have to wait for the lock
to be unlocked.
\item reader process locks the file for reading. Writers have to wait
for the lock, other readers can also lock the file for reading and read data.
\item in each moment there can be at most one active lock for writing or
multiple locks for reading. Both locks cannot be locked simultaneously.
\item for efficiency each process should hold the lock for shortest time
possible and if possible do not lock the whole file -- only the section that
is being worked with. Passive waiting is preferred, active waiting is
suitable for very short time.
\end{itemize}
\end{slide}

\begin{itemize}
\item 2 ways of waiting:
\begin{description}
\item[active (busy waiting)] -- a process tests a condition in a cycle while
it is not true.
\item[passive] -- a process registers itself in the kernel as a waiting for
the condition and goes asleep. The kernel wakes it up if the condition becomes
true.
\end{description}
\item \label{BUSYWAITING} active waiting is justifiable only in special
situations.
\end{itemize}

%%%%%

\pdfbookmark[1]{synchronization mechanisms}{synchromechs}

\begin{slide}
\sltitle{Synchronization mechanisms}
\begin{itemize}
\item theoretical solution -- mutual exclusion algorithms (Dekker
1965, Peterson 1981)
\item interrupt disable (1 CPU), special \emph{test-and-set} instructions
\item \emsl{lock-files}
\item tools offered by the operating system:
    \begin{itemize}
    \item \emsl{semaphores} (POSIX, System V IPC)
    \item \emsl{file level locking} (\texttt{fcntl()}, \texttt{flock()})
    \item thread synchronization: \emsl{mutexes} (surround critical sections,
    only one thread can hold the mutex), \emsl{condition variables}
    (the thread is blocked until another thread signals condition change)
    \emsl{read-write locks} (shared and exclusive locks, similar semantics
    to file level locking)
    \end{itemize}
\end{itemize}
\end{slide}

\begin{itemize}
\item Both Dekker and Peterson need to achieve the result via only shared
memory, i.e. several variables shared by processes.
\item Dekker's solution is presented as a first solution to the problem of
mutual exclusion of 2 processes, without having to apply the mechanism of
strict alternation, i.e. if second process does not express the will to enter
critical section, the first can enter how many times it wants
(and ifdef([[[NOSPELLCHECK]]], [[[vice versa]]])).
Dekker's solution is not trivial, compare it with 16 year younger
Peterson solution, e.g. on \texttt{en.wikipedia.org}.
\item We are not going to deal with theoretical algorithms or compare hardware
mechanisms used by the kernel. Instead we are going to focus on the use of
file level locking (which use the atomicity of some file operations)
and special synchronization primitives offered by the kernel.
\end{itemize}

%%%%%

\pdfbookmark[1]{lock files}{lockfiles}

\begin{slide}
\sltitle{Lock files}
\begin{itemize}
\item for each shared resource there exists previously agreed file path.
Locking is done by creating the file, unlocking by removing the file.
Each process must check if the file exists and if yes, has to wait.
\end{itemize}
\begin{alltt}
void \funnm{lock}(char *lockfile) \{
    while( (fd = open(lockfile,
                      O\_RDWR|O\_CREAT|\emprg{O\_EXCL}, 0600)) == -1)
        sleep(1); {\rm /* waiting in a loop for unlock */}
    close(fd);
\}

void \funnm{unlock}(char *lockfile) \{
    unlink(lockfile);
\}
\end{alltt}
\end{slide}

\begin{itemize}
\item \label{LOCK_UNLOCK} The key is the \texttt{O\_EXCL} flag.
\item Because the exclusive flag is not accessible in most shells, the shell
scripts usually use \texttt{mkdir} or \texttt{ln} (hard link) commands for
lock file synchronization.
\item example: \example{file-locking/lock-unlock.c} -- it is to be used together
with the \example{file-locking/run.sh} shell script.
\item In case of process crash the locks are not removed and therefore other
processes would wait forever. Thus it is prudent to write PID of the process
that created the lock to the lock file. The process that is waiting for unlock
can verify that the process with given PID number exists. If not, it can remove
the lock file and retry. User level command that can do this is e.g.
\emsl{\texttt{shlock}}(1) (on FreeBSD in \texttt{/usr/ports/sysutils/shlock}),
however could cause this situation:
\begin{itemize}
\item \emsl{watch out:} if multiple processes find out simultaneously that
the process does not exist, it can lead to error because the operation of
reading file contents and removing it is not atomic:
\begin{enumerate}
\item process A reads the contents of lock file, checks PID existence
\item process B reads the contents of lock file, checks PID existence
\item A deletes the lock file
\item A creates new lock file with its PID
\item B deletes the lock file
\item B creates new lock file with its PID
\item Now both processes think they acquired the lock.
\end{enumerate}
\end{itemize}
\item Another \emsl{problem} is that the \texttt{lock()} function above contains
active waiting.
This can be solved e.g. so that the process that acquired the lock can open
named pipe for writing. Reader processes will enter sleep by reading from the
pipe.
The \texttt{unlock()} will close the pipe and so the waiting processes will
be unblocked. See \example{pipe/countdown.c} for example with one reader.
\item Lock files are usually used for situations where multiple instances of
the same program are unwanted.
\end{itemize}

%%%%%

\pdfbookmark[1]{fcntl}{fcntl}

\begin{slide}
\sltitle{file locking: \texttt{fcntl()}}
\texttt{int \funnm{fcntl}(int \emph{fildes}, int \emph{cmd}, ...);}
\begin{itemize}
\item to set locks for file \texttt{fildes}:
\texttt{cmd}: 
    \begin{itemize}
    \item \texttt{F\_GETLK} \dots{} takes lock description from \nth{3} argument
    and replaces it with description of existing lock that collides with it.
    \item \texttt{F\_SETLK} \dots{} sets or destroys lock described by the \nth{3}
    argument. If the lock cannot be set immediately returns $-1$.
    \item \texttt{F\_SETLKW} \dots{} like \texttt{F\_SETLK}, however puts
    the process to sleep if it is not possible to set the lock 
    (\texttt{W} means ``wait'')
    \end{itemize}
\item \nth{3} argument contains lock description and is pointer to 
\texttt{struct flock}
\end{itemize}
\end{slide}

\begin{itemize}
\item Locking of files over NFS is done via the \texttt{lockd} daemon.
\item There are 2 types of locks:
    \begin{description}
    \item [advisory locks] -- for correct operation it is necessary that all
    processes working with locked files to check locks before reading/writing.
    used more frequently.
    \item [mandatory locks] -- if a file is locked, read/write operations will
    automatically block the process, i.e. the lock will be applied also on
    processes that do not explicitly work with the lock.
        \begin{itemize}
	\label{MANDATORY}
	\item not universally recommended, do not always work
        (e.g. \texttt{lockd} implements just advisory locking)
	\item for given file they are enabled by setting the SGID bit and
        removing right to execute for the group
	(i.e. setting that otherwise does not make sense).
	One process sets the lock (e.g. using \texttt{fcntl}). Other processes
	then do not have to check the lock explicitly because each
	\texttt{open/read/write} operation is checked by the kernel against
	the file locks and enforces waiting till the lock is explicitly
	unlocked by the owner process.\\
	\item implemented e.g. on Solaris, Linux. FreeBSD does not support it.
	The fcntl(2) man page on Linux (2013) does not recommend it because
	the implementation contains errors that could lead to race conditions
	and therefore the consistency cannot be generally achieved.
        \end{itemize}
    \end{description}
\item It is important to realize that when process exits, all its locks are
released.
\end{itemize}

%%%%%

\pdfbookmark[1]{flock}{flock}

\begin{slide}
\sltitle{File locking: \texttt{struct flock}}
\begin{itemize}
\item \texttt{l\_type} \dots{} lock type
    \begin{itemize}
    \item \texttt{F\_RDLCK} \dots{} shared lock (for reading)
    \item \texttt{F\_WRLCK} \dots{} exclusive lock (for writing)
    \item \texttt{F\_UNLCK} \dots{} unlock
    \end{itemize}
\item \texttt{l\_whence} \dots{} same as for \texttt{lseek()}, i.e.
\texttt{SEEK\_SET},
\texttt{SEEK\_CUR}, \texttt{SEEK\_END}
\item \texttt{l\_start} \dots{} start of locked region with regards to
\texttt{l\_whence}
\item \texttt{l\_len} \dots{} length of the region in bytes, 0 means till the
end of the file
\item \texttt{l\_pid} \dots{} PID of the process holding the lock, used
only for \texttt{F\_GETLK} when returning.
\end{itemize}
\end{slide}

\begin{itemize}
\item If given part is not locked when using \texttt{F\_GETLK},
the \texttt{flock} structure is returned unchanged except for the first member
that is set to \texttt{F\_UNLCK}.
\item \label{FCNTL_LOCKING} example: \example{file-locking/fcntl-locking.c}
\item \label{FCNTL_FIXED_RACE_C} example on how to use fcntl (it is 
,,fixed'' version of previous \example{race/race.c} example from page
\pageref{RACE_C}): \example{race/fcntl-fixed-race.c}.
\item locking via \texttt{fcntl} and \texttt{lockf} has one important property
that is being described in the \texttt{fcntl} man page in FreeBSD:

\emph{This interface follows the completely stupid semantics of System V
and IEEE Std 1003.1-1988 (``POSIX.1'') that require that all locks
associated with a file for a given process are removed when any file
descriptor for that file is closed by that process. This semantic
means that applications must be aware of any files that a subroutine
library may access. For example if an application for updating the
password file locks the password file database while making the
update, and then calls \texttt{getpwnam}(3) to retrieve a record,
the lock will be lost because \texttt{getpwnam}(3) opens, reads, and
closes the password database.}
\item The \texttt{lockf} function (SUSv3) is simpler variant of
\texttt{fcntl}, specifies only how to lock and how many bytes from the current
position in the file. Very often implemented as \texttt{fcntl} wrapper.
Beware: one cannot assume it is implemented around \texttt{fcntl} therefore
avoid using these both together.
\item The example \example{file-locking/lockf.c} demonstrates how mandatory
locking works and the use of the \texttt{lockf} function.
\end{itemize}


%%%%%

\pdfbookmark[1]{deadlock}{deadlock}

\begin{slide}
\sltitle{Deadlock}
\begin{itemize}
\item 2 shared resources \texttt{res1} and \texttt{res2} protected by locks
\texttt{lck1} and \texttt{lck2}. Processes \texttt{p1} and
\texttt{p2} want to have exclusive access to both resources.
\end{itemize}
\begin{center}
\input{img/tex/deadlock.tex}
\end{center}
\begin{itemize}
\item \emsl{watch out for the locking order !}
\end{itemize}
\end{slide}

\begin{itemize}
\item Generally deadlock happens if process is waiting for an event that cannot
happen. Here e.g. 2 processes are waiting for each other, for the other to
release the lock however that will never happen. Next possibility is deadlock
of single process that is reading from a pipe after forgetting to close
the write end of the pipe. If there is no one else that has the pipe open,
the read will block because the all file descriptor copies of the write end are
not closed and therefore the end of file cannot happen until the reading process
closed the write end however it cannot do it because it is blocked.
in \texttt{read} syscall:

\label{FIFODEADLOCK}
\begin{verbatim}
int main(void)
{
        int c, fd;
        mkfifo("test", 0666);
        fd = open("test", O_RDWR);
        read(fd, &c, sizeof(c));
        /* never reached */
        return (0);
}

$ ./a.out 
^C
\end{verbatim}
\item \texttt{fcntl()} checks for deadlock and returns \texttt{EDEADLK}.
\item It is best to avoid deadlock by correct programming and do not rely on the
system.
\end{itemize}


%%%%%

\pdfbookmark[1]{IPC}{sysvipc}

\begin{slide}
\sltitle{IPC}
\begin{itemize}
\item \emsl{IPC} stands for \emph{Inter-Process Communication}
\item between processes \emsl{in single system}, e.g. does not include
network communication.
\item \emsl{semaphores} \dots{} used for process synchronization
\item \emsl{shared memory} \dots{} passing data between processes,
brings similar problems as shared files, the solution is to use semaphores.
\item \emsl{message queues} \dots{} provide both communication and
synchronization (waiting for message to arrive) 
\item IPC means have \emsl{access rights} (for reading/writing) similarly to
files for owner, group and others.
\end{itemize}
\end{slide}

\begin{itemize}
\item IPC resources continue to exist even after the process that created them
is no longer around. To destroy them it is necessary to explicitly request this.
From the shell the list of IPS resources can be acquired using 
the \texttt{ipcs} command. They can be deleted using the \texttt{ipcrm} command.
The state and contents of existing IPS resources is unchanged even if no process
works with them at the moment.
\end{itemize}

%%%%%

\begin{slide}
\sltitle{Semaphores}
\begin{itemize}
\item introduced by E. Dijkstra
\item semaphore is data structure that contains:
    \begin{itemize}
    \item non-negative integer \texttt{i} (free capacity) 
    \item process queue \texttt{q}, that wait for free capacity
    \end{itemize}
\item semaphore operations: 
    \begin{description}
    \item [init(s, n)]~\\
    empty \texttt{s.q; s.i = n}
    \item [P(s)]~\\
    \texttt{if(s.i > 0) s.i-- else\\
    put calling process to sleep and add it to s.q }
    \item [V(s)]~\\
    \texttt{if(s.q empty) s.i++ else\\
    remove one process from s.q and wake it up}
    \end{description}
\end{itemize}
\end{slide}

\begin{itemize}
\item \emsl{P} is from dutch ifdef([[[NOSPELLCHECK]]],
[[[\uv{proberen te verlagen}]]]) -- try to
decrement, \emsl{V} from ifdef([[[NOSPELLCHECK]]], [[[\uv{verhogen}]]])
-- increment.
\item The \texttt{P(s)} and \texttt{V(s)} operations can be made generic:
the semaphore value is possible to change with any integer
\texttt{n} \dots{} \texttt{P(s,~n)}, \texttt{V(s,~n)}.
\item Allen B. Downey: \emph{The Little Book of Semaphores}, Second Edition,
on \url{http://greenteapress.com/semaphores/}
\item \emph{binary semaphore} has only values 0 or 1
\end{itemize}

%%%%%

\begin{slide}
\sltitle{Mutual exclusion with semaphores}
\begin{itemize}
\item one process initializes the semaphore
\begin{alltt}
sem s;
init(s, 1);
\end{alltt}
\item critical section is augmented with semaphore operations
\begin{alltt}
...
\emprg{P(s)};
critical section;
\emprg{V(s)};
...
\end{alltt}
\end{itemize}
\end{slide}


\begin{itemize}
\item semaphore initialized to \texttt{n} value allows \texttt{n} processes
to enter the critical section simultaneously. Here semaphore works like a lock.
The same process is unlocking (incrementing the value) and locking
(decrementing the value).
\item In general, the increment operation can be done by different process
than the one which performed the decrement operation (and
ifdef([[[NOSPELLCHECK]]], [[[vice versa]]])).
The mutexes work differently, see page \pageref{MUTEXES}.
\end{itemize}

%%%%%

ifdef([[[NOSPELLCHECK]]], [[[
\pdfbookmark[1]{sem\_open, sem\_wait, sam\_post, sem\_close}{posix-semaphores}
]]])

\label{NAMED_SEMAPHORES}
\begin{slide}
\sltitle{POSIX API for semaphores}

ifdef([[[NOSPELLCHECK]]], [[[
\begin{minipage}{\slidewidth}\vspace{-1\baselineskip}\texttt{\begin{tabbing}
sem\_t \funnm{sem\_open}(\=const *char \emph{name}, int \emph{oflag},
\\\>mode\_t \emph{mode}, unsigned int \emph{value});
\end{tabbing}}
\end{minipage}
]]])
\begin{itemize}
\item creates or opens a new POSIX semaphore.  \emph{mode} is same as for
\funnm{open}().  Use ``\texttt{/somename}'' for the \emph{name}.
\end{itemize}
ifdef([[[NOSPELLCHECK]]], [[[
\texttt{int \funnm{sem\_wait}(sem\_t *\emph{sem});}
]]])
\begin{itemize}
\item decrement \emph{\texttt{sem}}, if currently 0, the call will block
\end{itemize}
ifdef([[[NOSPELLCHECK]]], [[[
\texttt{int \funnm{sem\_post}(sem\_t *\emph{sem});}
]]])
\begin{itemize}
\item increment \emph{\texttt{sem}}. If \emph{\texttt{sem}} consequently becomes
greater than 0, another thread blocked in \funnm{sem\_wait}() will be woken up.
\end{itemize}
ifdef([[[NOSPELLCHECK]]], [[[
\texttt{int \funnm{sem\_close}(sem\_t *\emph{sem});}
]]])
\begin{itemize}
\item close \emph{\texttt{sem}}, free its resources allocated to \emsl{this}
process.
\end{itemize}
\end{slide}

\begin{itemize}
\item \emph{\texttt{oflag}} can be a combination of only \texttt{O\_CREAT} and
\texttt{O\_EXCL}.  If the semaphore already exists, an invocation with only
\texttt{O\_CREAT} silently succeeds, \texttt{O\_EXCL} will cause an error.
\item As soon as the semaphore is created, other processes can use it as
well, based on \emph{mode}.
\item If the semaphore is not removed via \funnm{sem\_unlink}(), it will exist
(in kernel) until the system is shut down.
\item There are also memory-based semaphores that do not use a name and are not
identified by the returned handle -- and thus use less resources.  See
\funnm{sem\_init}() and \funnm{sem\_destroy}() for more information.
\item Example on POSIX semaphores (it is a fixed version of the previous example
\example{race/race.c} from page \pageref{RACE_C}):
\example{race/posix-sem-fixed-race.c}.
\item To list POSIX semaphores system wide on Linux, list the contents of
\texttt{/dev/shm} pseudo file system (if mounted).
\end{itemize}

%%%%%

ifdef([[[NOSPELLCHECK]]], [[[
\pdfbookmark[1]{shm\_open, sem\_open, mq\_open}{posixipc}
]]])

\begin{slide}
\sltitle{Other IPC facilities}
\begin{itemize}
\item POSIX and SUSv4 define other facilities for interprocess communication we
have not mentioned yet:
    \begin{itemize}
    \item \emsl{POSIX shared memory} accessible via \funnm{shm\_open}()
    \item \emsl{POSIX queues} \dots{} \funnm{mq\_open}(),
    \funnm{mq\_send}(), \funnm{mq\_receive}(), \dots{} 
    \item \emsl{System V APIs} for queues, shared memory, and semaphores
    \end{itemize}
\item \emsl{sockets} come from the BSD world and allow communication in domains
\texttt{AF\_UNIX} (communication within the same system) and \texttt{AF\_INET}
(communication within the same system or across network).
\end{itemize}
\end{slide}

\begin{itemize}
\item POSIX IPC API is much simpler and better designed than the System~V IPC
API.  However, it is also much younger so you may see the System~V API used a
lot in existing code.  For System~V semaphore API and examples, see page
\pageref{SYSVSEM}.
\item The POSIX API for IPC came in with extension 1003.1b (aka POSIX.4), see
page \pageref{POSIX4}.
\item Sockets were accepted by other Unix systems as well and has been a part of
the UNIX specification since version 2.
\item There are other more system specific facilities.  For example,
\emph{doors} is an RPC like facility on Solaris, designed to be used within the
same system only. It has some interesting features - not only it can be used for
inter-process communication however it can be used in kernel to request a
service from a userland program.
\end{itemize}

\label{SYNCHRONIZATIONEND}

\endinput
