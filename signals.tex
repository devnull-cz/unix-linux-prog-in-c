
\pdfbookmark[0]{signals}{signals}
\slidecontents{5}

%%%%%%%%%%%%%%%%%%%%%%%%%%%%%%%%%%%%%%%%%%%%%%%%%%%%%%%%%%%%%%%%%%%%%%%%%

changequote([[[, ]]])

\begin{slide}
\sltitle{Introduction to signals}
\begin{itemize}
\item for notifying a process of asynchronous events, and for exception handling
\item mechanism of an interrupt available in user level
\item signal categories:
    \begin{itemize}
    \item \emsl{asynchronous events} happening independently of the main program
    flow, eg. a signal sent from another process, a timer expiration
    (\texttt{SIGALRM}), terminal disconnect (\texttt{SIGHUP}), or pressing
    \texttt{Ctrl-C} (\texttt{SIGINT})
    \item \emsl{exceptions} caused by a running process, eg.
    attempt to access a restricted area of memory (\texttt{SIGSEGV})
    \end{itemize}
\end{itemize}
\end{slide}

\label{SIGNALS}

\begin{itemize}
\item Interrupts can be viewed as a mean of communication between the CPU and
the OS kernel while signals are for communication between the kernel and
processes.
\item After returning from the handler, if that happens, the process continues
exactly from the place where the interruption happened.
\item Historically, signals were provided as a mechanism to forcefully terminate
processes.  That is why the function for sending signals is called
\funnm{kill}.
\item A nice example of an asynchronous event is while on Linux you send
a \texttt{SIGUSR1} to the \texttt{dd} command to print I/O statistics to
standard error output.  Start \texttt{dd} first, then send a couple of
\texttt{USR1} signals its way, like this:

\begin{verbatim}
$ kill -USR1 $(pgrep -f "dd if=/dev/zero of=/dev/null")
\end{verbatim}

You will see something like this:

\begin{verbatim}
$ dd if=/dev/zero of=/dev/null
9179287+0 records in
9179286+0 records out
4699794432 bytes (4.7 GB, 4.4 GiB) copied, 1.79083 s, 2.6 GB/s
14211424+0 records in
14211423+0 records out
7276248576 bytes (7.3 GB, 6.8 GiB) copied, 2.76889 s, 2.6 GB/s
\end{verbatim}
\end{itemize}

%%%%%

\begin{slide}
\sltitle{Introduction to signals (continued)}
\begin{itemize}
\item it is the simplest interprocess communication -- it only carries
information that an event happened
\begin{itemize}
\item there is a real-time POSIX extension that allows for more information,
more on that later
\end{itemize}
\item mostly processed asynchronously -- a signal interrupts the current process
flow and a \emph{signal handler} is invoked
\item signals are identified by numbers, represented by names like
\texttt{SIGSEGV}, \texttt{SIGCHLD}, or \texttt{SIGKILL}
\item name are usually \texttt{\#define}s, see \texttt{/usr/include/signal.h} or
\texttt{/usr/include/sys/signal.h}
\end{itemize}
\end{slide}

\begin{itemize}
\item The real-time extension is POSIX-1003.1b.
\pageref{REALTIMEEXTENSIONS}.
\item You can process signals in a synchronous way, see \funnm{sigwait}() on
page \pageref{SIGWAIT}.
\end{itemize}

%%%%%

\pdfbookmark[1]{kill}{kill}

\begin{slide}
\sltitle{Sending signals}
\texttt{int \funnm{kill}(pid\_t \emph{pid}, int \emph{sig});}
\begin{itemize}
\item sends signal \emph{sig} to a process or a process group
based on value of \emph{pid}: 
    \begin{itemize}
    \item \texttt{> 0} \dots{} to process with \emph{pid}
    \item \texttt{== 0} \dots{} to all processes in the same group
    \item \texttt{== -1} \dots{} to all aside from system processes
    \item \texttt{< -1} \dots{} to processes in a group \texttt{abs(pid)} 
    \end{itemize}
\item \texttt{sig == 0} means the system only checks whether the process has
enough privileges to send a signal without sending it
\item whether a process may send other process a signal depends on UID of both
processes
\end{itemize}
\end{slide}

\label{KILLSYSCALL}

\begin{itemize}
\item Traditionally, process with EUID~==~0 can send a signal to any other
process.  However, some systems optionally provide fine grain privileges and the
situation there is different even for root.  That is out of scope for this class
though.  \item Sending a signal to another process:
    \begin{itemize}
    \item Linux, Solaris: RUID or EUID of the process that sent the signal must
    match the real UID or saved SUID of the target process.
    \item FreeBSD: EUID of the processes must match
    \end{itemize}
\item Example: \example{signals/killing-myself.c}
\item 0 signal can be also used for a simple check for the specific process
existence, see \example{signals/check-existence.c}.
\end{itemize}

%%%%%

\pdfbookmark[1]{Handling signals}{sighandle}

\begin{slide}
\sltitle{Handling signals}
\begin{itemize}
\item unless a process sets it otherwise, each signal triggers a specific
default action, one of:
    \begin{itemize}
    \item terminate the process (\emsl{exit}) 
    \item terminate and dump a core (\emsl{core})
    \item ignore the signal (\emsl{ignore})
    \item stop the process (\emsl{stop})
    \item resume the process (\emsl{continue}) 
    \end{itemize}
\item process can either set to ignore a specific signal\dots
\item \dots{}or can handle the signal via a user-defined function,
called a \emsl{handler}
\end{itemize}

Signals \texttt{SIGKILL} and \texttt{SIGSTOP} \emsl{always} trigger an implicit
action, ie. exit or stop, respectively.
\end{slide}

\begin{itemize}
\item Creating a core dump means to store the contents of the process virtual
address space to a file.  Usually such file has a word \texttt{core} in its
filename.  Some systems, eg. macOS, may not generate core dumps by default even
if its for signals with default core dump action.
\item Most of the signals implicitly terminate the process, some create a core
dump on top of that to enable a post-mortem analysis.
\item The reason why \funnm{exec}() replaces all user set handlers to its
implicit action (see also page \pageref{EXEC}) is obvious -- code of the
original handlers no longer exists after the \funnm{exec}() call finishes.
\item You can learn the signal numbers and their names using the \texttt{-l}
option for the \texttt{kill(1)} command.  Without the argument, it will print
the list of all signals with their corresponding numbers.  Example:

\begin{verbatim}
$ kill -l SIGPIPE
13
\end{verbatim}
\item For each signal, you can learn what is its implicit action by checking a
manual page for function \funnm{signal}() or possibly for \emph{signal.h}.
\end{itemize}

%%%%%

\pdfbookmark[1]{Signal listing}{siglist}

\begin{slide}
\sltitle{Signal listing (1)}

We could divide signals into a few groups\dots

\emsl{Detected errors:}

\begin{tabular}{ll}
\texttt{SIGBUS} & bus error, eg. wrong alignment (core) \\
\texttt{SIGFPE} & floating point exception (core) \\
\texttt{SIGILL} & illegal instructions (core) \\
\texttt{SIGSEGV} & sigmentation violation (core) \\
\texttt{SIGPIPE} & write on a pipe with no reader (exit) \\
\texttt{SIGSYS} & non-existent system call invoked (core) \\
\texttt{SIGXCPU} & CPU time limit exceeded (core) \\
\texttt{SIGXFSZ} & file size limit exceeded (core)\\
\end{tabular}
\end{slide}

\begin{itemize}
\item Those signals are generated on an error in a program.
\item For the first four signals -- \texttt{SIGBUS}, \texttt{SIGFPE},
\texttt{SIGILL}, and \texttt{SIGSEGV} the standard does not specify what exactly
has to be the reason but usually those are errors detected by hardware.
Try the following examples, and check the return value with ``\texttt{kill -l
\$?}'': \example{signals/sigsegv.c},
\example{signals/div-by-zero.c}.
\item \label{SPECIALSIGNALS} \emsl{For those four signals, there are some
special rules as well} (for details, see section \emph{2.4 Signal Concepts} in
SUSv4):
\begin{itemize}
\item If set as ignored by function \funnm{sigaction}(), the program behavior
after such a signal is delivered is undefined.
\item The behavior of a process is undefined after it returns from a
signal-catching function.
\item If such a signal is masked while the signal is being delivered, the
program behavior is undefined.
\end{itemize}
\item Bottom line is that if a hardware generated error is real (ie. the signal
is not send via \funnm{kill} or similar functions), the process may never get
over that error at all.  It is not safe to ignore such errors, continue after
returning from a handler, or mask such signals.  You can catch such signals
though, the standard does not prohibit that.  However, if you do, you should do
so only to deal with the situation and exit.
You can check \example{signals/catch-SIGSEGV.c}. More information and another
example can be found on page \pageref{THREADS_SIGWAIT}.
\item Note: if the standard specifies any behavior as \emph{undefined}, it means
the specification does not state what should happend and that whatever happens
does not violate the standard.  So, if you trigger such a behavior and your
computer burns down or even flies off to the Moon, possibly still in flames,
that does not violate the standard either.
\end{itemize}

\begin{slide}
\sltitle{Signal listing (2)}

\emsl{Signals generated by a user or application:}

\begin{tabular}{ll}
\texttt{SIGABRT} & abort program (core) \\
\texttt{SIGHUP} & terminal line hangup (exit) \\
\texttt{SIGINT} & interrupt program via \texttt{Ctrl-C} (exit) \\
\texttt{SIGKILL} & kill program (exit, \emsl{cannot be caught or ignored})\\
\texttt{SIGQUIT} & quit program via \texttt{Ctrl-\bs} (core) \\
\texttt{SIGTERM} & software termination signal (exit) \\
\texttt{SIGUSR1} & user-defined signal 1 (exit) \\
\texttt{SIGUSR2} & user-defined signal 2 (exit) \\
\end{tabular}
\end{slide}

\begin{itemize}
\item Signal \texttt{SIGHUP} is often used as a way to let a daemon know its
configuration file changed and that it should re-read it.
\item \texttt{SIGINT} and \texttt{SIGQUIT} are usually generated from a terminal
via \texttt{Ctrl-C} and \texttt{Ctrl-\bs}, respectively, and could be redefined
using the command \texttt{stty} or via a function \funnm{tcsetattr}().  In order
to generate core files, your system must allow it.  Check command
\texttt{ulimit}.
\item \label{SIGKILL} As \texttt{SIGKILL} cannot be handled, use it only if you
know what you are doing.  For example, if a running process does not respond to
user input nor any other signal.  Many applications, mainly daemons, rely on the
fact that they are sent \texttt{SIGTERM} on termination, which they often handle
and perform pre-exit actions.  For example, flushing the database, removing
temporary files and file locks, etc.  So, do not use \texttt{SIGKILL} right away
only because "it's the simplest way to kill a process" as you might run in
trouble.
\item Example on using \texttt{SIGQUIT} on Solaris:

\begin{verbatim}
$ sleep 10
^\Quit (core dumped)
$ mdb core
Loading modules: [ libc.so.1 ld.so.1 ]
> $c
libc.so.1`__nanosleep+0x15(8047900, 8047908)
libc.so.1`sleep+0x35(a)
main+0xbc(2, 8047970, 804797c)
_start+0x7a(2, 8047a74, 8047a7a, 0, 8047a7d, 8047b91)
>
\end{verbatim}
\item \texttt{SIGTERM} is the default signal for command \texttt{kill(1)}.
\item \texttt{SIGUSR1} and \texttt{SIGUSR2} are not used by any system call and
are available for use to the user.
\end{itemize}

%%%%%

\begin{slide}
\sltitle{Signal listing (3)}

\emsl{Job control:}

\begin{tabular}{ll}
\texttt{SIGCHLD} & child status has changed (ignore)\\
\texttt{SIGCONT} & continue after stop (continue) \\
\texttt{SIGSTOP} & stop (stop; \emsl{cannot be caught or ignored}) \\
\texttt{SIGTSTP} & stop from terminal \texttt{Ctrl-Z} (stop) \\
\texttt{SIGTTIN} & background read attempted from control terminal (stop) \\
\texttt{SIGTTOU} & background write attempted to control terminal (stop) \\
\end{tabular}

\end{slide}

\begin{itemize}
\item Those used to be part of a non-mandatory POSIX extension but now they are
required by POSIX.1-2008.  See also page \pageref{UNIXSTANDARDS}.
\item Only one process at any given time can read from its process group control
terminal but multiple processes can write to it at the same time.
\item Stopping a process group from a terminal, usually via \texttt{Ctrl-Z}, is
done with signal \texttt{SIGTSTP}, not \texttt{SIGSTOP}; a program thus can
catch the signal. \label{SIGTSTP}
\end{itemize}

%%%%%

\begin{slide}
\sltitle{Signal listing (4)}

\emsl{Timers:}

\begin{tabular}{ll}
\texttt{SIGALRM} & timer expired (exit) \\
\texttt{SIGPROF} & profiling timer alarm (exit; see \funnm{setitimer}(2)) \\
\texttt{SIGVTALRM} & virtual time alarm (exit; also see \funnm{setitimer}(2)) \\
\end{tabular}

\emsl{Miscelaneous:}

\begin{tabular}{ll}
\texttt{SIGPOLL} & event occured on explicitly watched file descriptor (exit) \\
\texttt{SIGTRAP} & trace trap (core) \\
\texttt{SIGURG} & urgent condition present on socket (ignore) \\

\end{tabular}
\end{slide}

\begin{itemize}
\item \texttt{SIGALRM} and related function \funnm{alarm} is used for setting
timer alarms, useful for implementation of timeouts, for example.
\end{itemize}

%%%%%

\pdfbookmark[1]{sigaction}{sigaction}

\begin{slide}
\sltitle{Setting actions for signals}
\begin{minipage}{\slidewidth}\vspace{-1\baselineskip}\texttt{\begin{tabbing}
int \funnm{sigaction}(\=int \emph{sig},
const struct sigaction *\emph{act},\\\> struct sigaction *\emph{oact});
\end{tabbing}}
\end{minipage}
\begin{itemize}
\item assigns an action \emph{act} for a signal \emph{sig},
previous setting is returned in \emph{oact}, if non-zero
\item structure \texttt{sigaction} contains:
    \begin{itemize}
    \item \texttt{void (*\emph{sa\_handler})(int)} \dots{} \texttt{SIG\_DFL},
    \texttt{SIG\_IGN}, or a handler address (ie. a handler function name)
    \item \texttt{sigset\_t \emph{sa\_mask}} \dots{} signals blocked while in
    handler; plus \emph{sig} is blocked by default
    \item \texttt{int \emph{sa\_flags}} \dots{} \texttt{SA\_RESETHAND} (handler
    is reset to \texttt{SIG\_DFL} at the moment the signal is delivered),
    \texttt{SA\_RESTART} (restart pending calls), \texttt{SA\_NODEFER}
    (do not block \emph{sig} while in handler)
    \end{itemize}
\end{itemize}
\end{slide}

\begin{itemize}
\item If \texttt{act == NULL} you only get the present setting and nothing is
changed.  If not interested in previous setting, use \texttt{NULL} for
\emph{oact}.
\item If \texttt{SA\_RESTART} is not set, the system call interrupted by the
signal will return an error with \texttt{errno} set to \texttt{EINTR} (some
calls actually return directly the \texttt{errno} value but that is not relevant
now).  However, resetting the call will not work for all system calls.  You
should consult the documentation for your system if you intend to use it.
On Linux, check the \texttt{signal(7)} manual page.  Example:
\example{signals/interrupted-read.c}.
\item Beware of deadlocks (will be discussed in the chapter on synchronization).
If you set \texttt{SA\_NODEFER}, your handler needs to be reentrant.
\item \label{ASYNCSIGNALSAFE} \emsl{You should only use functions in a safe way
from a handler}.  By safe it is meant either use reentrant functions or you need
to make sure the signal is not delivered in the wrong time (for example, the
signal is delivered within a function and the same function is called from the
handler, and the function is not ready for that).  Minimal set of functions
that must be \emsl{\emph{async-signal-safe}} is listed in SUSv4 in section
\emph{System Interfaces: General Information $\Rightarrow$ Signal Concepts
$\Rightarrow$ Signal Actions (2.4.3)}.  Systems can extend the list, of course.
Whether a function is safe to use in a signal handler or not should be
documented in its manual page.
\item Using static data and/or locking in a function would generally be a
problem for its asynchronous signal safety.  It can lead to corrupt data,
deadlocks, etc.  As the set of functions safe to use in a handler is limited,
one way to use the signals is only to set a global variable in the handler and
test it later, for example in a loop that processes some events.  As a function
waiting for an event is typically interruptible by a signal (see above), setting
the global variable and testing it later does not experience any real delay.
Example: \example{signals/event-loop.c}.
\item There is also a simplified signal facility \funnm{signal}().  We recommend
to use \funnm{sigaction}() only.  Behavior of \funnm{signal}() is not specified
with threads, for example.  It can also behave in a very different way on
different systems, some keep the handler set after delivering the signal, some
reset it to \texttt{SIG\_DFL}.  Check \example{signals/signal-vs-sigaction.c} if
interested.
\item \label{REALTIMEEXTENSIONS} If your system supports a part of POSIX.1b
called \emph{Realtime Signals Extension} (RTS), it is possible to use the
extension if you use flag \texttt{SA\_SIGINFO}.  In that case a
\texttt{sa\_sigaction} member of the structure \texttt{sigaction} must be used
for the handler, not \texttt{sa\_handler}.  The new handler has three parameters
and it is possible to learn the PID of a signalling process, its UID, and more.
See manual page \texttt{signal.h(3HEAD)} on Solaris, the online SUS
specification of the header file, or the book [POSIX.4], page
\pageref{REF_PROGRAMMING}.  More information is also on page \pageref{POSIX} and
\pageref{SIGWAITINFO}.  Examples: \example{sig\-nals/siginfo.c},
\example{sig\-nals/sigqueue.c}.
\end{itemize}

%%%%%

\begin{slide}
\sltitle{Example: setting time limit for read}
\setlength{\baselineskip}{0.8\baselineskip}
\begin{alltt}
void handler(int sig)
\{ write(2," !!! TIMEOUT !!! \bs{}n", 17); \}

int main(void) 
\{
    char buf[1024]; struct sigaction act; int sz;
    act.sa\_handler = handler;
    \emprg{sigemptyset}(&act.sa\_mask);
    act.sa\_flags = 0;
    \emprg{sigaction}(SIGALRM, &act, NULL);
    \emprg{alarm}(5);
    sz = read(0, buf, 1024);
    alarm(0);
    if (sz > 0)
        write(1, buf, sz);
    return (0);
\}
\end{alltt}
\end{slide}

\label{SIGALRM}

\begin{itemize}
\item You can also use timers with a finer granularity provided by
\funnm{setitimer}() and \funnm{getitimer}() functions.  After the timer expires,
a process is sent a signal based on the value of the first argument
\emph{which}:
\begin{itemize}
\item \texttt{ITIMER\_REAL} \dots{} timer decrements in real time, and
\texttt{SIGALRM} is used.
\item \texttt{ITIMER\_VIRTUAL} \dots{} timer decrements in process virtual time
which runs only when the process is executing.  A signal \texttt{SIGVTALRM} is
used.
\item \texttt{ITIMER\_PROF} \dots{} timer decrements both in process virtual
time and when the system is running on behalf of the process, and
\texttt{SIGPROF} is used.
\end{itemize}
\item Note that we use \funnm{write}() system call to write the text.  Functions
like \funnm{printf}() are generally not safe to use in signal handlers, see page
\pageref{ASYNCSIGNALSAFE}.
\item Example: \example{signals/alarm.c}
\end{itemize}

%%%%%

\pdfbookmark[1]{sigprocmask, sigpending}{sigblock}

\begin{slide}
\sltitle{Signal blocking}
\begin{itemize}
\item blocked signals are delivered to the process after getting unblocked again
\end{itemize}
\begin{minipage}{\slidewidth}\texttt{\begin{tabbing}
int \funnm{sigprocmask}(\=int \emph{how}, const sigset\_t *\emph{set},\\
\>sigset\_t *\emph{oset});
\end{tabbing}}
\end{minipage}
\begin{itemize}
\item sets the blocked signal mask and returns the old one
\item \emph{how} -- \texttt{SIG\_BLOCK} adds signals to block,
\texttt{SIG\_UNBLOCK} for removing signals, for a complete mask change
use \texttt{SIG\_SETMASK}
\item for manipulating signal sets, use \funnm{sigaddset}(),
\funnm{sigdelset}(), \funnm{sigemptyset}(), \funnm{sigfillset}(),
\funnm{sigismember}()
\end{itemize}
\texttt{int \funnm{sigpending}(sigset\_t *\emph{set});}
\begin{itemize}
\item returns a mask for the signals pending for delivery
\end{itemize}
\end{slide}

\label{SIGPROCMASK}

\begin{itemize}
\item The system silently ignores attempts to block \texttt{KILL} and
\texttt{STOP}.
\item It is a difference to ignore and block a signal.  A signal ignored will
never be delivered to the process, while a blocked signal is remembered by the
kernel and is delivered if the process unblocks it.
\item It is implementation dependend if a signal received multiple times while
blocked will be delivered only once or multiple times after unblocking.
\item If POSIX.4 extension is used (see page
\pageref{REALTIMEEXTENSIONS} and the use of \texttt{SA\_SIGINFO}), signals are
delivered through a queue and multiple signals are never squeezed into one
delivery.
\item Both arguments \emph{oset} and \emph{set} may be \texttt{NULL}.  If both
are \texttt{NULL} in the same invocation of the function, nothing happens.
\end{itemize}

%%%%%

\begin{slide}
\sltitle{Example: blocking signals}
\setlength{\baselineskip}{0.8\baselineskip}
\begin{alltt}
sigset\_t sigs, osigs; struct sigaction sa;
sigfillset(&sigs); \emprg{sigprocmask}(SIG\_BLOCK, &sigs, &osigs);
switch(cpid = fork()) \{
    case -1:
        \emprg{sigprocmask}(SIG\_SETMASK, &osigs, NULL);
        ...
    case 0: /* Child */
        sa.sa\_handler = h\_cld; sigemptyset(&sa.sa\_mask);
        sa.sa\_flags = 0;
        \emprg{sigaction}(SIGINT, &sa, NULL);
        \emprg{sigprocmask}(SIG\_SETMASK, &osigs, NULL);
        ...
    default: /* Parent */
        \emprg{sigprocmask}(SIG\_SETMASK, &osigs, NULL);
        ...
\}
\end{alltt}
\end{slide}

\begin{itemize}
\item \label{SIGNALBLOCKINGEXAMPLE} The example shows a situation when a child
needs to use a different signal handler from its parent.  If you just changed
the handler in the child without blocking the signals first in the parent you
would create a potential race -- a child would for a short time run with an
unwanted handler.  Note that \funnm{fork}() does not change the signal mask, see
page \pageref{FORK}.
\item Out of simplicity we block all the signals in the example even that on
pages \pageref{SPECIALSIGNALS} and \pageref{THREADS_SIGWAIT} it is explained why
it is not a good thing.
\end{itemize}

%%%%%

\label{SIGWAIT}
\pdfbookmark[1]{pause, sigsuspend, sigwait}{sigwait}

\begin{slide}
\sltitle{Waiting for a signal}
\texttt{int \funnm{pause}(void);}
\begin{itemize}
\item suspends the caller until delivery of a signal that is not blocked or
ignored
\end{itemize}
\texttt{int \funnm{sigsuspend}(const sigset\_t *\emph{sigmask});}
\begin{itemize}
\item like \funnm{pause}(), also temporarily changes the blocked signal mask
\end{itemize}
\texttt{int \funnm{sigwait}(const sigset\_t *\emph{set}, int *\emph{sig});}
\begin{itemize}
\item waits for a signal from \emph{set} (must be blocked) and puts the signal
number to \emph{sig}.
\item the signal handler is not called
\end{itemize}
\end{slide}

\begin{itemize}
\item When in \funnm{pause}() or \funnm{sigsuspend}(), a signal not blocked will
call a handler, if one is set, and the signal catching function returns, unless
the process is terminated by the signal.
\item With \funnm{sigwait}(), you can implement a synchronnous signal
handling.  First, a process blocks signals it is interested in, then when
convenient, it waits for them or just checks whether there are some pending
using \funnm{sigpending}().
\item \label{SIGWAIT} Funtion \funnm{sigwait}() was added with POSIX-1003.1c
extensions (threads) and you should use this function when dealing with signals
in the multithreaded application.  As other POSIX thread functions, it returns
an error number directly, and 0 on success.
\item \label{SIGWAITINFO} There are also similar functions \funnm{sigwaitinfo}()
and \funnm{sigtimedwait}(), defined with the POSIX-1003.1\emsl{b} extension
(real-time).  They work in a similar way but set \texttt{errno} on failure and
it is possible to get more information upon signal delivery using a structure
\texttt{siginfo\_t}, see page \pageref{REALTIMEEXTENSIONS}.
\item Example (a signal is used for synchronization of two processes
communicating via a shared memory): \example{signals/sigwait.c}
\item Do not use for synchronous signals like \texttt{SIGSEGV}, \texttt{SIGILL},
etc.  More on that on pages \pageref{SPECIALSIGNALS} and
\pageref{THREADS_SIGWAIT}. 
\end{itemize}

\endinput
